\documentclass[sigconf,10pt]{acmart}

\usepackage{booktabs} % For formal tables
\usepackage{listings}

\lstset{
  basicstyle=\tiny,
  frame=single,
  tabsize=2.
  language=c++,
  numbers=left
}

% Copyright
%\setcopyright{none}
%\setcopyright{acmcopyright}
%\setcopyright{acmlicensed}
\setcopyright{rightsretained}
%\setcopyright{usgov}
%\setcopyright{usgovmixed}
%\setcopyright{cagov}
%\setcopyright{cagovmixed}


% DOI
%\acmDOI{10.475/123_4}

% ISBN
%\acmISBN{123-4567-24-567/08/06}

%Conference
\acmConference[PACT'19]{ACM International Conference on Parallel Architectures and Compilation Techniques}{September 2019}{Seattle, WA, USA}
\acmYear{2019}
%\copyrightyear{2016}


%\acmArticle{4}
%\acmPrice{15.00}

% These commands are optional
%\acmBooktitle{Transactions of the ACM Woodstock conference}
%\editor{Jennifer B. Sartor}
%\editor{Theo D'Hondt}
%\editor{Wolfgang De Meuter}


\begin{document}
\title{Software Restoration for Legacy-Parallel Applications}
%\titlenote{Software Restoration}
%\subtitle{Extended Abstract}
%\subtitlenote{The full version of the author's guide is available as
%  \texttt{acmart.pdf} document}


\author{Vladimir Janjic, Christopher Brown and Adam Barwell}
%\authornote{Dr.~Trovato insisted his name be first.}
%\orcid{1234-5678-9012}
\affiliation{%
  \institution{School of Computer Science, University of St Andrews}
  \city{St Andrews}
  \state{UK}
  %\postcode{43017-6221}
}
\email{adb23,cmb21,vj32@st-andrews.ac.uk}

% The default list of authors is too long for headers.
\renewcommand{\shortauthors}{V. Janjic et al.}


\begin{abstract}
    \emph{Parallel patterns} enable developing structured parallel programs that are maintainable, adaptive and portable while also achieving good performance on a variety of parallel systems. However, there still exists a large base of \emph{legacy-parallel code} which was developed using low-level parallel libraries such as \emph{pthreads} which would benefit from structured parallelism. In this paper, we present the \emph{software restoration} methodology for rewriting legacy-parallel applications into structured parallel code using parallel patterns. We also describe software \emph{refactorings} to eliminate ad-hoc (pthread) parallelism from the legacy-parallel code, which is a first step in the proposed methodology of software restoration. Finally, we demonstrate the benefit of software restoration on a number of realistic benchmarks and use-cases in terms of gained performance, increased adaptivity, portability and maintainability.
\end{abstract}

%
% The code below should be generated by the tool at
% http://dl.acm.org/ccs.cfm
% Please copy and paste the code instead of the example below.
%
\begin{CCSXML}
<ccs2012>
 <concept>
  <concept_id>10010520.10010553.10010562</concept_id>
  <concept_desc>Computer systems organization~Embedded systems</concept_desc>
  <concept_significance>500</concept_significance>
 </concept>
 <concept>
  <concept_id>10010520.10010575.10010755</concept_id>
  <concept_desc>Computer systems organization~Redundancy</concept_desc>
  <concept_significance>300</concept_significance>
 </concept>
 <concept>
  <concept_id>10010520.10010553.10010554</concept_id>
  <concept_desc>Computer systems organization~Robotics</concept_desc>
  <concept_significance>100</concept_significance>
 </concept>
 <concept>
  <concept_id>10003033.10003083.10003095</concept_id>
  <concept_desc>Networks~Network reliability</concept_desc>
  <concept_significance>100</concept_significance>
 </concept>
</ccs2012>
\end{CCSXML}

\ccsdesc[500]{Computer systems organization~Embedded systems}
\ccsdesc[300]{Computer systems organization~Redundancy}
\ccsdesc{Computer systems organization~Robotics}
\ccsdesc[100]{Networks~Network reliability}


\keywords{Parallel patterns, refactoring, code analysis}


\maketitle

\input{pact2019-body}

\bibliographystyle{ACM-Reference-Format}
\bibliography{pact2019}

\end{document}
